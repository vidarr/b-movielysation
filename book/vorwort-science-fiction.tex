\subsection{Science-Fiction}

Wer Science Fiction hört, erwartet Geschichten aus einer Zukunft, mal näher, mal ferner der Gegenwart.
Diese Zukunft zeigt sich dabei im Film immer durch fortschrittliche Technik und Technologie in der Filmwelt.
Die Zuordnung gestaltet sich oft willkürlich, da kaum ``reine'' Science-Fictionwerke auftreten.
% Prägenstes Genre Zombiehorror, aber auch science fiction vertreten
Ein berrühmt-berüchtigtes Beispiel hierfür liegt mit ``Plan 9 from outer Space'' vor.
In diesem aberwitzigen Machwerk des Antimeisters Ed Wood versuchen Außerirdische, die Erde zu erobern.
Dass sie dabei auf den namensgebenden ``Plan 9'' zugreifen, und also Wiedergänger zur Eroberung zu nutzen, stellt diesen Film auch in eine Reihe mit dem klassischen Zombiehorror ``Night of the Living Dead'', und so könnte man auch eine Einordnung in eben dieses Genre diskutieren.
Seines Namens zum Trotz finden sich in Science-Fictionwerken immer wieder starke Fantasyelemente, so in ``Star Troopers''.
Die apokalyptische Welt, so unvermögend sie auch erschaffen ist, lebt vom Religiösen, ja die ganze Geschichte fußt stark auf der Geburt eines ultimativen Dämons durch die Bestie Ragnarok, auch wenn dessen Schaffen sich im Wesentlichen auf ``geboren werden'', ``Pissen'' und ``gepfählt werden''\footnote{Ein für einen Weltenvernichter bemerkenswerter Lebensweg, der sich innerhalb weniger Minuten entrollt.} beschränkt.
Auffällig ist die anscheinende Vorliebe von japanischen Filmemachern für die Kombination von Science Fiction mit Splatterelementen, wie sich in ``Alien vs. Ninja'' genauso wie in ``Meatball Machine'' zeigt.
Dass Science Fiction in Billigproduktionen auf ganz besondere Probleme stößt, kommt nicht von ungefähr.
Die vom Publikum oft erwarteten futuristischen, aufwändigen Effekte und herausfordernde Szenen in den Weiten des Weltraumes leiden oft verständlicherweise besonders unter den eng gefassten Budgetgrenzen.
Ein Paradebeispiel dafür, wie desaströs Filmteams an dieser Hürde scheitern können, findet sich im Film ``Dünyayı Kurtaran Adam'', den man einfach gesehen haben muss, da zumindest mein Sprachvermögen in keinster Weise hinreicht, den Film in verdienter Weise in Worte zu fassen.
Ob dies der Grund ist, warum die vorliegende Sammlung nur eine kleine Auswahl an Science Fiction enthält, muss dahingestellt bleiben.
