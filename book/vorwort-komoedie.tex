\subsection{Komödie}

\paragraph{Was}
ist Komödie? 
Dieser Frage lässt sich auf mannigfaltige Weise begegnen.
Man kann Komödie als rein sprachlichen Begriff sehen, und sich ihm nähern, indem man im durch die Geschichte nachspürt.
\paragraph{Tut}
man dies, soweit uns dies heute noch möglich ist, gelangt man wie so oft schließlich zur griechischen Antike, man ist geneigt zu seufzen, ,,schon wieder''.
Neben Wein, philosophischer und orgiastischer Auschweifung, Wahlrecht und Knabenliebe und dem umweltfreundlichen Schiffsantrieb wurde im Schatten mediterraner Sonne eine ausgeprägte Theaterkultur geboren. 
Vielleicht dienten sie ursprünglich kultischen Zwecken, wären dann wohl zunächst eher ernster Natur gewesen. 
Wenn es so gewesen sein sollte, schaffte es die Schauspielerei aber irgendwann, sich hiervon zu emanzipieren. 
Schließlich dienten Theateraufführungen nicht zuletzt der Unterhaltung und brachten damit neben ernsten Inhalten auch leichtverdauliche, heitere Stücke hervor.
Irgendwann kamen für beide Varianten Namen auf, ,,Tragödie'' und ,,Komödie'' waren geboren, jedenfalls die Begriffe.
Man wird diesen Begriffen vermutlich nicht lückenlos bis in die Gegenwart verfolgen können, doch kann man in der Geschichte wiederholt Spuren derselben finden, welche zum Verständnis desselben beitragen können.
\paragraph{Besonders}
heraus sticht ,,Komödie'' als Begriff bei Dante Alighieris ,,Commedia''  auf, später ,,Divina Commedia'' genannt.
Das ist im Hinblick auf den Inhalt des Werkes interessant, handelt das Werk doch von einer spirituellen Reise durch die jenseitigen Welten, ein Zeugnis tiefer religiöser Überzeugung des Autors und seiner Zeit, offensichtlich ein Ausdruck starker Gefühle.
Beladen mit Symbolik nutzt Dante sein Werk auch, um Kritik zu üben, indem er etwa auch einen Papst in der Hölle oder den damaligen Kaiser Heinrich VII. im höchsten Himmelskreis darstellt.
Unterhaltung ist offensichtlich kein Ziel dieses Werkes.
Immerhin, das Werk endet glücklich, und dies ist wohl auch, was ,,Commedia'' zumindest im Hinblick auf dieses Werk ausdrücken soll.
Offenbar definiert sich Komödie zu dieser Zeit wohl zuerst als das Gegenstück zur Tragödie, die ein unglückliches Ende besitzt.
\paragraph{Ganz} 
offenbar allerdings ist die Bedeutung dort nicht stehengeblieben. 
Ein glückliches Ende reicht heute nicht mehr aus, um ein Werk als Komödie auszuzeichnen, soviel ist klar. 
Die Welt ist wie so oft, komplizierter geworden.
Längst haben sich unzählige weitere Genres etabliert, Thriller können glückliche Enden ebenso wie Actionwerke oder Liebesfilme besitzen, ohne auch Komödien zu sein.
Was heute eine Komödie zu einer Komödie macht, ist also mehr.
Nach dem Ausschlusskriterium könnte man herangehen und allem das Etikett Komödie vorenthalten, was die Stimmung der Zuschauer drückt. 
Doch was sollte man dann unter eine Tragikkomödie verstehen?
Man wird schließlich aber nicht umhin kommen, auch Tragikkomödien Momente wenn nicht des Glücks so doch der Heiterkeit zuzugestehen.
Doch ist ein Drama, das alleine durch ungewollt stümperhafte Dialoge das Publikum zum Schmunzeln zwingt, gleich eine Tragikkomödie?
Wie verhält es sich, wenn der Autor Dialoge bewusst stümperhaft inszeniert, um eben ein Schmunzeln zu provozieren? 
Und ist eine Komödie auch dann noch eine Komödie, wenn das Schmunzeln beim Publikum ausbleibt, der Schöpfer in seiner Bemühung die Zuschauer zu erheitern, scheitert?
\paragraph{Doch} ist dies wirklich ein Nähern? Ist es nicht vielmehr ein Davonlaufen, da man der Frage im Grunde nur ausweichen will, weil man Angst davor hat, dass die eigentliche Antwort erst einmal ausgesprochen ihre ganze Banalität enthüllt und man sich eingestehen muss: Da ist nichts Erhabenes, nichts Ewiges, keine tiefere Weisheit in dieser Antwort?
Mag sein - allerdings ist dies auch kein Kirchenschiff und dieser Text keine rebellische Predigt, die durch vermoderte Hallen schallt. \footnote{Dies ist ja auch nicht der Ort, sondern ein Weg.}
Von daher hinfort mit Gravitas und zurück zu der Frage, die uns hier eigentlich umtreibt, zurück zur Komödie.
Ein weiterer Weg, sich dem Wesen von Komödie zu nähern, kann dahin eingeschlagen werden, dass man es als das begreift, was es ist, es seiner geschichtlichen Bürde entkleidet und quasi die Quintessenz herausdestilliert, den barocken Aus- und Abschweifungen des immer weltfremden Bildungsbürgertums und seiner bis ins Undurchdringliche stilisierten, gekünstelten und auf Verschleierung angelegten Sprache mit ihren Windungen und Wirrungen entsagt, und mit einfachen Worten einen klaren Punkt setzt: 
Komödie ist, was gemacht wurde, um zum Lachen anzuregen. 
Für unsere Zwecke sollte diese ,,Definition'' jedenfalls allemal hinreichen:
Eine Komödie ist ein Film, der mit der Absicht produziert wurde, Heiterkeit und Gelächter auszulösen, es spielt dabei keine Rolle, ob der Film dieses tatsächlich leistet.
\paragraph{Ein}
Film, der witzig sein will, ist eine Komödie.
Ein Film, der witzig sein will, es aber nicht schafft, ist eine schlechte Komödie.
Ein Film, der nicht witzig sein will, es aber dennoch ist, ist keine Komödie. \footnote{Vermutlich aber auch schlecht}
Die Frage, inwieweit ,,schlechte'' Filme gute Komödien zu sein vermögen, mögen die folgenden Seiten beantworten.
 

%% Absicht entscheidend, nicht Erfolg (auch nichtlustige Komödien möglich)


