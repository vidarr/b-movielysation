\subsection{Vorwort}

\paragraph{Am Anfang} war das \emph{BIIING}.
Und das \emph{BIIING} war nicht Wort, sondern Klang.
Und der Klang kehrte wieder. Und wieder. Und wieder.
Oft in Begleitung einer Maske, immer unerwartet, stets unverstanden.
Am Anfang war der Zombie-Holocaust.
Und er war schlecht, wirklich schlecht. 
Natürlich ist hier die Rede von einem Film, doch was für ein Film.
Der Titel lässt schon vermuten, dass diesem Film auf absehbare Zeit die Würdigung als wertvoller Kulturbeitrag verwehrt bleiben dürfte.
Im Gründungsmythos des Projektes kommt diesem Film die zentrale Rolle zu.
Einst war eine Zeit, da nichts war, bis ein Schüler dieses Filmes angesichtig wurde.
Und er sah, dass er schlecht war. Außerordentlich schlecht sogar, von so herausstechender Schlechtheit, die ihn dazü nötigte, diese Filmerfahrung in Worte zu kleiden, und der Welt, zumindest dem Teil, der sich auf seine Website verirren würde, Kund zu tun.
Das Projekt war geboren.
Dieser Film war der Zünder, die Keimzelle, der Urfilm, das Alpha, der Nukleus der Sammlung des schlechten Geschmacks, deren Hüter sich hier zu Wort melden.
Dieser Film ist einer unter vielen, 
es gibt zahllose Filme wie diesen.
Und jeder Film ist doch mehr.
Er ist ein Symbol, die Scheide zwischen den Welten der Mainstreamunterhaltung und der schillernden Welt des Unvermögens.
Man kann ihn aber auch als Prüfung begreifen.
Wer die Willenskraft aufbringt, diesem Film in Gänze zu folgen, den Verlockungen des Entschlummerns oder des Sich-entnervt-Abwendens trotzt, der hat die wichtigste Hürde auf dem Weg durch den Spiegel getan, der ist bereit, dass sich ihm die Welt der zweit-, drittklassigen Unterhaltung öffnet, eine Welt voller Bizzarheit, vielleicht nicht Wundern, aber zumindest Absonderlichkeiten, und hin und wieder kleinen Perlen, eine Welt über die sich genauso staunen wie lachen lässt, ein, auf seine ganz eigene Weise, Wunderland.
Zugegeben, dieses Wunderland ist versteckt und unter Mengen von Morast begraben, Filmen, deren Handlung sich im ausufernden Gedärmfontänen erschöpft, Filmen mit Mikrofonen im oder vielleicht auch ganz ohne Bild, Filmen, die so fade sind, dass sie nicht mal mehr nerven, Filmen, bei deren teuerstem Requisit es sich um die dem Zuschauer geraubte Zeit handelt.
Und so muss man sich den Zutritt hart und immer wieder aufs Neue erarbeiten, das Leben ist hart, die Ausbeute nicht regelmäßig und man sieht einem Film nicht an, ob der billige Umschlagdruck Ausdruck erquicklicher technischer Inkompetenz oder einfach kühler harter Kostenkalkulation ist.
Aber die Perlen sind da, sofern man bereit ist, sein Bewusstsein zu erweitern, wenn man bereit ist, Stümperei und Geschmacklosigkeit als Tugend zu erkennen.

Diese Schrift ist der Versuch eines Berichts, den aktuellen Stand unserer Reise durch diese Welt, mal abstoßend, mal faszinierend, aber immer eine Reise wert.

