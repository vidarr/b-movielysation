\subsection{Kunstfilm}

Es gibt Fragen, die viele Leute intuitiv vermeiden, vielleicht rein aus
Instinkt, vielleicht aus Erfahrung. 
Es gibt Fragen, die das Potential besitzen, Freundschaften zu zerstören, Ehen zu
brechen und Chaos und Verwirrung unter die Völker zu tragen. 
Oft besitzen diese Fragen eine einfache Sache, die das Verhängnis bringt. Diese
Sache ist: Keine Antwort. 
Der Klassiker unter den Grollbringern kommt in den unschuldigen Worten: 
"Gibt es einen Gott, und wenn ja, welcher Diät hängt er gerade an?"
Weniger verheerend, aber ebenso heimtückisch entpuppt sich die Frage, was Kunst
sei. 
Wir werden hier also das einzig Richtige tun und gar nicht erst nach der Antwort
suchen. 
Andere Strategien existieren, führen in der Regel unvorsichtig behandelt auch
nur in das Dunkel. 
"Kunst liegt im Auge des Betrachters" - eine bei Licht betrachtet eine
Nebelkerze, sechs Worte, erfunden um um die Leere zu füllen, die sich auftut,
blickt man unter den Kunstbegriff, eine Leere, die endlos ist. 
Und so kann sich Gott und die Welt als Kunst verstehen, ohne diesen hohlen
Begriff zu füllen.
So wenig Substanz "Kunst" als Begriff besitzt, so verbreitet scheint die
Vorstellung, dass dieser aus unerfindlichen Gründen nicht mit dem Begriff
"B-Movie" vereinbar sei. 
Dabei tummeln sich gerade hier eine unüberblickbare Masse an Machwerken, die
ganz den Untugenden des Kunstbegriffes frönend Durcheinander und Chaos in einer schier
unüberschaubaren Vielfalt zu beschwören scheinen.
Folgerichtig finden sich in dieser Kategorie allen voran Werke, die dies zu
immer neuen Höhen treiben.
Dies wirkt wie ein weiterer Versuch, dem Kunstbegriff beizukommen. 
Tatsächlich könnte nichts ferner liegen, könnte man doch auch schlicht
umformulieren: 
Hier finden sich alle Filme, die irgendie nirgends anders
hingepasst haben.
Weithin unerreicht steht hier \emph{Mein Traum oder die Einsamkeit ist nie
allein}.


