\subsection{Kunstfilm}

Es gibt Fragen, die viele Leute intuitiv vermeiden, vielleicht rein aus
Instinkt, vielleicht aus Erfahrung. 
Es gibt Fragen, die das Potential besitzen, Freundschaften zu zerstören, Ehen zu
brechen und Chaos und Verwirrung unter die Völker zu tragen. 
Oft besitzen diese Fragen eine einfache Sache, die das Verhängnis bringt. Diese
Sache ist: Keine Antwort. 
Der Klassiker unter den Grollbringern kommt in den unschuldigen Worten: 
"Gibt es einen Gott, und wenn ja, welcher Diät hängt er gerade an?"
Weniger verheerend, aber ebenso heimtückisch entpuppt sich die Frage, was Kunst
sei. 
Wir werden hier also das einzig Richtige tun und gar nicht erst nach der Antwort
suchen. 
Andere Strategien existieren, führen in der Regel unvorsichtig behandelt auch
nur in das Dunkel. 
"Kunst liegt im Auge des Betrachters" - eine bei Licht betrachtet eine
Nebelkerze, sechs Worte, erfunden um um die Leere zu füllen, die sich auftut,
blickt man unter den Kunstbegriff, eine Leere, die endlos ist. 
Und so kann sich Gott und die Welt als Kunst verstehen, ohne diesen hohlen
Begriff zu füllen.
So wenig Substanz "Kunst" als Begriff besitzt, so verbreitet scheint die
Vorstellung, dass dieser aus unerfindlichen Gründen nicht mit dem Begriff
"B-Movie" vereinbar sei. 
Dabei tummeln sich gerade hier eine unüberblickbare Masse an Machwerken, die
ganz den Untugenden des Kunstbegriffes frönend Durcheinander und Chaos in einer schier
unüberschaubaren Vielfalt zu beschwören scheinen.
Folgerichtig finden sich in dieser Kategorie allen voran Werke, die dies zu
immer neuen Höhen treiben.
Dies wirkt wie ein weiterer Versuch, dem Kunstbegriff beizukommen. 
Tatsächlich könnte nichts ferner liegen, könnte man doch auch schlicht
umformulieren: 
Hier finden sich alle Filme, die irgendie nirgends anders
hingepasst haben.
Weithin unerreicht steht hier \emph{Mein Traum oder die Einsamkeit ist nie
allein}.
Vordergründig folgt man einer bizarren nächtlichen Reise eines
Allerweltsmenschen. Nachdem er einer Frau, die sich "Godot" nennt, wird er von
ihr durch Kanalisation und morbide Fernsehshows mit Hitler aus der Mülltonne
geführt. 
Hintergründig offenbart sich - ja was eigentlich?
Ebenfalls einer Reise folgt man in Izo, wenn es hier auch der blutige Pfad eines Samurai
ist, der sich nicht nur von Ort zu Ort, sondern auch von Zeitalter zu Zeitalter
metzelt und schließlich sogar den Tod herausfordert.
Viele Fragen und wenig Antworten gibt es auch bei "Daniel der Zauberer", einem
der wenigen bleibenden Spuren die Daniel Küblbeck auf dieser Welt bisher
hinterlassen hat. 
Und irgendwie kann was man hier zu sehen bekommt, unmöglich alles sein, was der
Film zu bieten hat - dieser Film \emph{muss} einfach eine verborgene Ebene
besitzen, gut verborgen zugegeben, und wenn man die offensichtliche Ebene
gesehen hat, kann man sich nur wünschen, dass die verborgene genau das für immer bleibt.
\emph{Pontypool} dagegen besticht durch seinen Minimalismus. 
Filme über Zombieinvasionen gibt es wie Sand am Meer, aber wenige, in dem
man das Tun eines Radiomoderators in seinem Studio verfolgt, während er über
eben jene Invasion berichtet und bis kurz vor Ende überhaupt kein Zombie auftritt.
Schwer glaublich, dass diese Idee funktioniert, hier jedoch tut sie das überaus
gut, und lässt viel Diskussionsbedarf aufkommen.
Das schaffen auch \emph{Mann beißt Hund} und \emph{Menschenfeind}, allerdings mit geradezu 
widerwärtiger Brutalität, die sogar durch endlose Gedärmfontänen abgestumpfe
Horrorfans das Zusehen zur Qual macht. 
Dazu beitragen mag die völlig andere Qualität ihrer Gewaltxzesse, die weniger
blutig, dafür sehr viel stärker auf die Psyche schlagen.
Es mag ironisch anmuten, dass ausgerechnet Gewaltexzesse in der Sphäre der
B-Movies also dazu taugen, zum Kunstfilm zu qualifizieren.
Gerade diese innere scheinbare Zerrissenheit, die Variabilität und Gewissheit,
im Vorhinein nie zu wissen, was einen erwartet, die Kerntugenden eines B-Movies, 
sind hier in diesem Genre auf die Spitze getrieben.
Gerade dieses dem Gipfel nachjagen, ihn immer höher legen zu wollen, macht
dieses Kategorie besonders verlockend.
Denn eines ist sicher: Zu reden hat man niemals so viel wie nach dem Genuss
eines B-Movie-Kunstfilms.
 

Kunstfilme sind ihrem Charakter nach oft nicht ganz einfach zu konsumieren.



